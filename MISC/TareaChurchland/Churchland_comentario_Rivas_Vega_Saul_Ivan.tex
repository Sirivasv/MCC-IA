\documentclass[12pt]{article}
\usepackage[utf8]{inputenc}
\usepackage{array}
\newcolumntype{C}[1]{>{\centering\let\newline\\\arraybackslash\hspace{0pt}}m{#1}}
\usepackage[spanish]{babel}
 \usepackage{url}
\usepackage[spanish, fixlanguage]{babelbib}
\bibliographystyle{IEEEtran}
\usepackage{graphicx}
\graphicspath{ {./images/} }
\usepackage{amssymb}
\usepackage{amsmath}
\usepackage{algorithm2e}

\title{Comentario crítico sobre el artículo:\\ “Could a Machine Think?” \\ de Paul M. Churchland y Patricia Smith Churchland}

\author{
	Saul Ivan Rivas Vega \\
	\\
	Inteligencia Artificial\\
\\
}

\date{\today}

\begin{document}
	\maketitle
	\pagebreak
	\paragraph{}
	En este artículo se habla sobre ideas muy interesantes que envuelven el trabajo de la inteligencia artificial, en una primera instancia definen la pregunta, considero esto bastante importante porque tenemos un punto de partida y las ideas o suposiciones posteriores deben estar acorde con la pregunta, ¿Podría una maquina, que manipula símbolos físicos siguiendo reglas sensibles a la estructura, pensar? Así no estamos hablando de una máquina cualquiera.\\
	Por otra parte habla de la prueba de Turing, pero da la siguiente afirmación, la restricción de que la comunicación en la prueba fuera por medio de una interacción \textit{tele-typed} (por escrito en máquina de escribir, tecleado) era innecesaria si la función que produce como salida el comportamiento de una persona consciente es computable, esto por que una maquina que manipula símbolos que la compute puede interactuar con el mundo de maneras mas complejas con visión directa o con el habla.\\
	Para mi eso extiende las capacidades que se pretende lograr, el tener toda esa interacción requieren distintos componentes que no necesariamente tienen que ver con el pensamiento, no es necesaria la restricción si queremos replicar lo mejor posible una persona, sin embargo el contexto que da origen a la prueba es el pensamiento y como será revisado posteriormente, esto implica a discriminar los elementos relevantes y los que no en la modelación de la mente.\\
	\\
	Revisemos el ejemplo de John Searle, consiste en que una máquina que tenga un nivel conversacional en Chino tal que pase la prueba de Turing y además que la estructura interna sea tal que no importa como se comporte, un observador se mantiene seguro de que ni la máquina o cualquier parte de ella entiende Chino. En el artículo mencionan que hubo ciertas diferencias con lo expuesto por Searle, aunque no estaban de acuerdo con algunas exponen que todas rechazan uno de los axiomas presentados por Searle, “La sintaxis por si sola no es constitutiva o suficiente para la semántica” para posteriormente rechazar la conclusión, “Los programas no son constitutivos o suficientes para las mentes”. Parte de su rechazo es la analogía de hacer las mismas observaciones en los descubrimientos de las propiedades de la luminiscencia. Y mencionan que aunque estén de acuerdo con Searle de que las expectativas de cognición artificial son muy pobres hacen la diferencia explicando que ellos lo hacen de un punto de comportamiento mostrado por las máquinas no desde una posición de intuiciones de sentido común acerca de la presencia o la ausencia de contenido semántico.
	\\
	Mi opinión acerca de esto parte de la simple definición de \textit{entendimiento}, se asume que ni los componentes ni la máquina entiende Chino, sin embargo a que se refieren con entender, si es interpretarlo y responder pues claro que lo entiende por que es lo único que hace, por otro lado el entendimiento tiene una extensión con el significado de las cosas, la máquina no lo entiende por que solo esta siguiendo las reglas sin tener consciencia de lo que esta haciendo y lo que significa. Pero ahora llegamos a que es la consciencia, o si esta es requerida para el entendimiento y no existe una idea universalmente aceptada sobre la consciencia y la evaluación de que significa tenerla.\\\\ Concuerdo de que es erróneo pensar que por un mínimo de escepticismo a la verdadera aportación de los elementos que constituyen un sistema, este en conjunto sea falso. 
	Como en otros artículos se habla de lo que es relevante o no, pienso que aquí yace mucho de las discusiones a lo largo de la historia y actualidad de la inteligencia artificial, que tanto del poder causal del cerebro debemos replicar, si su composición al mas mínimo detalle, si su reacción a otros químicos o su olor. Llegamos a que no todo es necesario para cumplir el objetivo de crear una mente artificial así como un avión no requiere dejar huevos para volar. 
	\\
	El cerebro, el órgano que se dio a si mismo un nombre sigue sin ser descifrado en su totalidad, y con la revisión de los puntos tratados en el artículo como la ingeniería inversa para explicar la cognición concluyo diciendo que la búsqueda de replicar su funcionamiento en una estructura artificial seguirá basándose en los nuevos descubrimientos que deje la investigación introspectiva sobre el cerebro y el pensamiento. 
	
 \end{document}
